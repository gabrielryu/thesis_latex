\chapter{Introduction}
\label{cha:introduction}

%\chapterquote{I'm awesome!}{Barney Stinson, WIRED magazine, 19.1.2009}

Most legacy network systems are composed of centralized architecture, in which the main system affords services to connected users. When users request any information, the front-end application takes over requests of the user and brings data from the connected server or back-end application. This process takes advantage on the development and management. However, as the growing scale of the system, the centralized network encounters difficulties, such as network connection instability, DDoS, storage cost with rising scale, and sensitive data management. In contrast to the centralized network system, the decentralized network system has become a big trend to avoid the disadvantages of the centralized network. Distributed Ledger Technology (DLT) is one of the decentralized network technologies. It is composed of nodes that get verified with the cryptographical identifier and then interconnect using the peer-to-peer network, and transmit data between nodes. The system to avoid data modification uses consensus algorithms for reliable nodes. The decentralized storage to share data uses the distributed file system, such as InterPlanetary File System \cite{IPFS}\cite{benet2014ipfs}, BitTorrent File System (BTFS) \cite{BTFS}, Ethereum Swarm, and BigchainDB. The distributed file systems are characterized by sharing fragmented data as a chunk that is stored in nodes. It relies on distributed hash tables (DHT) \cite{sivaraja2008efficient}, a lookup service using key and value or list paired. Using encrypted key, DHT can get exact values. With distinguishing DHT characters, more use cases are extended (e.g., e-Wallet, Digital Identification, Distributed File System, and Peer-to-Peer file sharing). DHT based P2P system uses algorithms, such as Chord, CAN, Pastry, Tapestry, and Kademlia. Kademlia has a binary tree structure and uses the XOR metric to calculate the distance between nodes. The characteristic of Kademlia is that between nodes communicate using a UDP-based transmission. Thus, the current UDP-based Kademlia system restricts on storing and sharing a large amount of data. Many different consensus algorithms, DHT, and databases have emerged to meet the challenges of distributed systems.

The target scenario is the implementation using Kademlia DHT-based P2P network system \cite{maymounkov2002kademlia} that supports to transmit and store the large data. When the data owner uploads the data in the Kademlia DHT P2P network, the peer is created, including the data. As another user wants to download the data, the owner attempts to connect the peer and then download the data. The peer is laid on the network topology using the XOR metric-based deterministic distance method. A peer can retrieve another peer’s location using the computed distance. This method gives the benefits of giving fast search and reducing attempts. 

However, while communicating among peers, there is a problem. The transmittable size of a packet should be under the MTU, which is 1500 bytes. The large size packet over the MTU limitation will be blocked in the filtering at the Firewall or NAT router. The packet can be lost in transmission. Moreover, the UDP protocol does not support the control of the transmission, so the peer cannot guarantee received data. The TCP protocol has a transmission control service. Therefore, when it detects that there is lack of a packet, a receiver requests a sender to retransmit the lack of packet. These procedures lead on increasing traffic problems and traffic congestions. Because of this reason, we consider a novel protocol or alternative method to distributed traffic. 

We planned a new model to resolve the issues. The essential Kademlia-based DHT P2P network system is provided from T-Lab. The existing system will be updated to transmit and store large data. The new model will be implemented using the direct tunnel for the data transmission. Furthermore, the implementation supports a novel transport protocol, which is QUIC. This protocol is expected to improve data transmission. Also, it offers a connection identifier for mobility devices. It is useful to reconnect the host. If the device allocates a new IP address, the device does not need to reconnect to the host. The host can recognize it using the record of the connection ID. Both models should be cross-compiled. The implementation should be feasible to connect between the PC and IoT device or IoT devices. Resource-constrained devices are the target for this implementation.

I will introduce the simple, relevant background knowledge, problems, the purpose of this thesis, requirements, and plans in this section. Section 2 presents the related work of P2P networks, Kademlia DHT, and transport protocols. The background knowledge will help the thesis to be understood. Section 3 explains the concept and design of the implementations. Section 4 gives a detailed description of the implementation. Proposals of both models will be tested and measured. Section 5 shows the results of the evaluation. Finally, Section 6 summarizes and concludes the achievements from this study and gives implications for future work.
