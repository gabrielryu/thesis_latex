\chapter{Conclusion}
\label{cha:conclusion}

At present, there are various devices connected in networks. The devices are required to be connected to a distributed network rather than a traditional centralized network, depending on their functionalities and purposes. The decentralized P2P networks are robust and easier to scale. Therefore, it is a good alternative for resource-constrained devices such as IoT devices. P2P networks can be divided into unstructured and structured networks. The unstructured networks have peers randomly laid on the network without the specific criteria and communication with each other (e.g., BitTorrent). On the other hand, structured networks are logically composed with specific methods. This thesis focuses on this structure.

The Staxnet is a structured P2P network system. The peer is located using the Kademlia DHT methodology and communicates with each other. They communicate by sending and receiving data, but there are restrictions on data transmission. The data transmission is under the MTU limitation. Because of this limitation, communication has the disadvantage of making large data transmission. In order to figure out this problem, other studies have given several suggestions by using jumbo frame, PLMTUD, and traffic distribution methods. 

Therefore, this thesis focuses on how efficient the large data transmit and store in the Kademlia DHT network. We planned the direct tunnel-based P2P model to resolve the MTU problem and to improve the performance of the data transmission. In the research, a novel transport protocol, which is QUIC, can resolve the problems of existing major protocols. It provides fast connectivity, packet error correction, and connection identifier to reduce reconnection for mobile devices. Therefore, we designed the planned model with this protocol.

This model is implemented on the provided Staxnet network. The Staxnet consists of two main parts as daemon and CLI application. We added an additional module, such as the mapping server in the daemon, that manages peers in the DHT network and holds on to the operation method. The file transmission is designed to be switched to the CLI application. Also, we applied the new QUIC protocol to improve transmission performance. In comparison with other common protocols, which are TCP and UDP, the CLI application is implemented to support using them. The clients can select the protocols to transmit data. The existing system is implemented to allow large file transmission. In the key-value pair, the value is used for the data space.

We evaluated the existing model and the implemented new model. The evaluation measured the data transmission time compared to both models and each protocol. The test uses four different file sizes using 1 Megabyte, 3 Megabytes, 5 Megabytes, 10 Megabytes. The existing model shows that as the file size increased, the time that transmission took also increased accordingly. We found that in the results, the upload time of this model takes twice as long as the download time. We expect that the peer in this model reads the whole file as it has visited another peer. Because of this reason, overload occurs in the upload. The direct tunnel-based P2P model shows significant results. The large file transmission in the download reduces the transmission time compared to the existing model. More specifically, download time using QUIC is faster than other protocols in 10 Megabytes sharing (being about 70\%); the QUIC protocol takes 0.156 seconds, the TCP protocol takes 0.609 seconds, and the UDP protocol takes 0.586 seconds. We tested the data transmission experimentally using the different MTU (being 1500 bytes and 5000 bytes). The data transfer rate using the bigger MTU is about 30\% faster. It cannot utilize in real-world networks because router and gate-ways restricted the MTU size; it is traditionally set about 1500 bytes. When a packet over the MTU limitation is attempted to pass the router, which sets 1500 bytes of the MTU, it is blocked. We can recognize that QUIC protocol has the potential for advancing network performance.

\section{Future works}

To follow up on such thesis questions, the QUIC protocol keeps working for the HTTP mapping over QUIC for HTTP/3 standard. Therefore, it has added new functionalities and subtracted existing functionalities. This protocol is not stable yet. When it is completely implemented, the Kademlia DHT can apply this protocol. The essential Kademlia DHT supports only the UDP protocol. However, it is possible to use this protocol because this protocol has similar features compared to UDP. The advantages of QUIC gives more benefit, which are to improve data transmission, give more powerful secure method, and give mobility. 

The Staxnet system uses the memorial DB. If it deploys other DB systems, the performance of data storage can be improved. It is suited for the small data handling in the DHT network because this system is developed for resource-constrained devices. Hence, it lacks performance in large data handling. When the large data are processing on the system, occasionally, memory error occurred. The error handling needs to be improved. 
