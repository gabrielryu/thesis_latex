\chapter*{Zusammenfassung}
\label{cha:zusammenfassung}

Die netzwerkverbindungsbasierten Geräte entwickeln sich schnell in verschiedenen Bereichen, abhängig von ihrer Leistung, Mobilität und ihrem Zweck. In einem dezentralen Netzwerk müssen sich die Geräte direkt im eingesetzten Netzwerk und über verschiedene Netzwerke hinweg miteinander verbinden und kommunizieren. Die gängigsten Netzwerklösungen sind jedoch die Server-Client-basierten, zentralisierten Architekturen, die zwar einfacher einzurichten sind, aber in Bezug auf Ausfallsicherheit und Skalierbarkeit Schwächen aufweisen können. Eine Alternative sind dezentrale Peer-to-Peer (P2P)-Netzwerke, die robust und einfacher zu skalieren sind. Sie erfordern jedoch aufwändigere Protokolle und entsprechende Technologien. Diese Besonderheiten müssen für die Vorteile aufgelöst werden. Das Staxnet ist ein Kademlia DHT (Distributed Hash Table) basiertes P2P-Netzwerksystem. Die Netzwerktopologie besteht aus einer deterministischen XOR-Metrik-basierten Distanz entsprechend des Kademlia DHTs. Dieser Kademlia Algorithmus hat den Vorteil, dass er eine schnelle dezentrale Peer-Suche ermöglicht. Allerdings hat die Kommunikation zwischen Peers einen Nachteil. Bei der Datenübertragung kann nur ein Paket übertragen werden, das kleiner ist als die Größe der MTU (Maximum Transmission Unit). Der Hauptzweck dieser Arbeit ist es, zu sehen, wie effizient große Daten (> MTU) im Kademlia DHT-Netzwerk übertragen werden.

Daher wird in dieser Arbeit das direkte tunnel-basierte P2P-Modell vorgeschlagen. Wenn ein Peer die übertragbare Datei im Kademlia DHT P2P-Netzwerk besitzt, fordert ein anderer Peer die Datei an, um sie herunterzuladen. Beide Peers bauen einen direkten Tunnel auf und übertragen die Datei über den aufgebauten Tunnel. Außerdem wird bei der Datenübertragung ein neuartiges QUIC-Transportprotokoll für die Übertragung großer Datenmengen verwendet. Die Diplomarbeit evaluiert das bestehende System und dieses Modell unter Verwendung verschiedener Transportprotokolle für die Datenübertragung. Die Ergebnisse der Evaluierung zeigen, dass das auf einem direkten Tunnel basierende P2P-Modell gut abschneidet und die Datenübertragung unter Verwendung des QUIC-Protokolls 0,133 Sekunden für den Download einer 11-Megabyte-Datei benötigt. Es ist schneller als andere Modelle mit dem TCP-Protokoll (0,702 Sekunden) oder dem UDP-Protokoll (0,559 Sekunden). Das Modell, das das QUIC-Protokoll verwendet, zeigt eine bessere Leistung als die anderen. Allerdings gibt es bei diesem Modell noch Punkte, die berücksichtigt und verbessert werden müssen.

Hier kommt das deutsche Abstract hin. Wie das geht, kann man wie immer auf Wikipedia nachlesen \url{http://de.wikipedia.org/wiki/Abstract}...