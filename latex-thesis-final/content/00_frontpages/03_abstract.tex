\chapter*{Abstract}
\label{cha:abstract}

The network connect-based devices are rapidly developing in various fields depending on their performance, mobility, and purpose. In a decentralized network the devices are required to interconnect and communicate with each other directly on the deployed network and across different networks. However, the most well-known networks are the server-client based centralized architectures, which are easier to set up but may lack in terms of resilience and scalability. Alternatives are decentralized peer-to-peer (P2P) networks, which are robust and easier to scale. However, it requires complex protocols and relevant technologies. These features should be resolve for the benefits.The network connect-based devices are rapidly developing in various fields depending on their performance, mobility, and purpose. In a decentralized network the devices are required to interconnect and communicate with each other directly on the deployed network and across different networks. However, the most well-known networks are the server-client based centralized architectures, which are easier to set up but may lack in terms of resilience and scalability. Alternatives are decentralized peer-to-peer (P2P) networks, which are robust and easier to scale. However, it requires complex protocols and relevant technologies. These features should be resolve for the benefits.
 
The Staxnet is Kademlia DHT (Distributed Hash Table) based P2P network system. The network topology is composed of XOR metric-based deterministic distance according to Kademlia DHT. This algorithm gives an advantage of giving fast peer search. However, the communication between peers has a drawback. The data transmission should transmit a packet smaller than the size of MTU (Maximum Transmission Unit). The main purpose of this thesis is to see how efficient the large data transmits in the Kademlia DHT network.

Therefore, this thesis proposes the direct tunnel-based P2P model. When a peer has the transmittable file in the Kademlia DHT P2P network, another peer requests the file to download it. Both peers make a direct tunnel and transfer the file via the built tunnel. Moreover, the data transmission applies a novel QUIC transport protocol for large data transfer. The thesis evaluates the existing system and this model using different transport protocols in the data transmission. The results of the evaluation show that the direct tunnel based P2P model has performed well and the data transmission using the QUIC protocol takes 0.156 seconds to download 10 Megabytes file. It is faster than other models using the TCP protocol (0.609 seconds) and the UDP protocol (0.586 seconds). The model using the QUIC protocol shows better performance than others. However, this model still leaves points that need to be considered and to be improved.
